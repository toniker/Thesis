\documentclass[a4paper,nobib,justified]{tufte-book}

\usepackage[utf8]{inputenc}
\input{preamble.tex}

% Greek
\usepackage[greek,english]{babel}
\usepackage{alphabeta}
\let\textlozenge\undefined
\usepackage{gfsartemisia-euler}
\NewDocumentCommand{\g}{+m}{\foreignlanguage{greek}{#1}}
\NewDocumentCommand{\e}{+m}{\foreignlanguage{english}{#1}}
%\usepackage{hyphenation-greek}
%\input{hyphenation-el.tex}
\DefineBibliographyStrings{english}{%
	page             = {σ\adddot},
	pages            = {σσ\adddot},
}
\DefineBibliographyExtras{USenglish}{%
	% d-m-y format for long dates
	\protected\def\mkbibdatelong#1#2#3{%
		\iffieldundef{#3}
		{}
		{\stripzeros{\thefield{#3}}%
			\iffieldundef{#2}{}{\nobreakspace}}%
		\iffieldundef{#2}
		{}
		{\mkbibmonth{\thefield{#2}}%
			\iffieldundef{#1}{}{\space}}%
		\iffieldbibstring{#1}{\bibstring{\thefield{#1}}}{\stripzeros{\thefield{#1}}}}%
	% d-m-y format for short dates
	\protected\def\mkbibdateshort#1#2#3{%
		\iffieldundef{#3}
		{}
		{\mkdatezeros{\thefield{#3}}%
			\iffieldundef{#2}{}{/}}%
		\iffieldundef{#2}
		{}
		{\mkdatezeros{\thefield{#2}}%
			\iffieldundef{#1}{}{/}}%
		\iffieldbibstring{#1}{\bibstring{\thefield{#1}}}{\mkdatezeros{\thefield{#1}}}}%
}

% Override tufte command for headers to remove greek accents
\makeatletter
\ExplSyntaxOn
\cs_new_protected:Npn \removeaccent #1 {
	\tl_set:Nn \replace_str {#1}
	\def\acctonos{}
	\let\accdialytikatonos\accdialytika
	\tl_replace_all:Nnn \replace_str { ά } { α }
	\tl_replace_all:Nnn \replace_str { έ } { ε }
	\tl_replace_all:Nnn \replace_str { ή } { η }
	\tl_replace_all:Nnn \replace_str { ί } { ι }
	\tl_replace_all:Nnn \replace_str { ό } { ο }
	\tl_replace_all:Nnn \replace_str { ύ } { υ }
	\tl_replace_all:Nnn \replace_str { ώ } { ω }
	\tl_replace_all:Nnn \replace_str { ϊ } { ϊ }
	\tl_replace_all:Nnn \replace_str { ϋ } { ϋ }
	\tl_replace_all:Nnn \replace_str { ΐ } { Ϊ }
	\tl_replace_all:Nnn \replace_str { ΰ } { Ϋ }
	\tl_replace_all:Nnn \replace_str { Ά } { α }
	\tl_replace_all:Nnn \replace_str { Έ } { Ε }
	\tl_replace_all:Nnn \replace_str { Ή } { Η }
	\tl_replace_all:Nnn \replace_str { Ί } { Ι }
	\tl_replace_all:Nnn \replace_str { Ό } { Ο }
	\tl_replace_all:Nnn \replace_str { Ύ } { Υ }
	\tl_replace_all:Nnn \replace_str { Ώ } { Ω }
	%    \tl_replace_all:Nnn \replace_str { Ϊ } { Ι }
	%    \tl_replace_all:Nnn \replace_str { Ϋ } { Υ }
	\tl_use:N \replace_str
}
\ExplSyntaxOff
\renewcommand\mainmatter{%
	\if@openright%
	\cleardoublepage%
	\else%
	\clearpage%
	\fi%
	\@mainmattertrue%
	\fancyhf{}%
	\ifthenelse{\boolean{@tufte@twoside}}%
	{% two-side
		\renewcommand{\chaptermark}[1]{\markboth{##1}{}}%
		\fancyhead[LE]{\thepage\quad\smallcaps{\newlinetospace{\removeaccent{\plaintitle}}}}% book title
		\fancyhead[RO]{\smallcaps{\newlinetospace{\removeaccent{\leftmark}}}\quad\thepage}% chapter title
	}%
	{% one-side
		\fancyhead[RE,RO]{\smallcaps{\newlinetospace{\removeaccent{\plaintitle}}}\quad\thepage}% book title
	}%
}
\makeatother


%%
% Book metadata
%\title{Design of Fault Detection, Isolation and Recovery in the AcubeSAT nanosatellite\thanks{AcubeSAT}}
\title{Επικοινωνία υποσυστημάτων στον νανοδορυφόρο AcubeSAT}
\author[Αντώνιος Κερεμίδης]{Αντώνιος Κερεμίδης}
\publisher{\ensuregreek{Αριστοτελειο Πανεπιστημιο Θεσσαλονικης}}

\hypersetup{
	pdftitle={Επικοινωνία υποσυστημάτων στον νανοδορυφόρο AcubeSAT},
	pdfsubject={Επικοινωνία υποσυστημάτων στον νανοδορυφόρο AcubeSAT},
	pdfauthor={Αντώνιος Κερεμίδης},
	addtopdfcreator={tufte-book class}
}

\DeclareAcronym{service}{
  long = υπηρεσία ,
  tag = glossary , no-index
}
\DeclareAcronym{parameter}{
  short = parameter,
  long = παράμετρος ,
  tag = glossary , no-index
}
\DeclareAcronym{microcontroller}{
  long = {μικροελεγκτής} ,
  tag = glossary , no-index
}
\DeclareAcronym{requirement}{
	long = {προδιαγραφή} ,
	long-plural-form = {προδιαγραφές} ,
	tag = glossary , no-index
}
\DeclareAcronym{verification}{
	long = {επαλήθευση} ,
	tag = glossary , no-index
}
\DeclareAcronym{standard}{
	long = {πρότυπο} ,
	tag = glossary , no-index
}
\DeclareAcronym{interface}{
	long = {διεπαφή} ,
	tag = glossary , no-index
}
\DeclareAcronym{bus}{
	long = {δίαυλος} ,
	tag = glossary , no-index
}
\DeclareAcronym{enumeration}{
	long = {απαρίθμηση} ,
	tag = glossary , no-index
}
\DeclareAcronym{driver}{
	long = {οδηγός περιφερειακού} ,
	tag = glossary , no-index
}
\DeclareAcronym{cold redundancy}{
	long = {ψυχρός πλεονασμός} ,
	extra = {passive redundancy},
	tag = glossary , no-index
}
\DeclareAcronym{warm redundancy}{
	long = {θερμός πλεονασμός} ,
	tag = glossary , no-index
}
\DeclareAcronym{hot redundancy}{
	long = {ενεργός πλεονασμός} ,
	extra = {active redundancy},
	tag = glossary , no-index
}

\begin{document}

%\renewcommand*{\itemautorefname}{Στοιχείο}
%\renewcommand*{\figureautorefname}{Σχήμα}
%\renewcommand*{\chapterautorefname}{Κεφάλαιο}
%\renewcommand*{\sectionautorefname}{Ενότητα}
%\renewcommand*{\subsectionautorefname}{Ενότητα}
%\renewcommand*{\tableautorefname}{Πίνακα}
%\newcommand*{\algorithmautorefname}{Αλγόριθμος}
%\renewcommand{\appendixpagename}{Παραρτήματα}
%\newcommand{\Παράρτημαautorefname}{Παράρτημα}

\renewcommand*{\figurename}{Σχήμα}
\renewcommand*{\tablename}{Πίνακας}
\renewcommand*{\contentsname}{Περιεχόμενα}
\renewcommand*{\listfigurename}{Κατάλογος Σχημάτων}
\renewcommand*{\listtablename}{Κατάλογος Πινάκων}
\sisetup{range-phrase={ ως }}

\renewcommand{\crefpairconjunction}{ και\nobreakspace}%
\renewcommand{\creflastconjunction}{ και\nobreakspace}%
\renewcommand{\crefpairgroupconjunction}{ και\nobreakspace}%
\renewcommand{\creflastgroupconjunction}{, και\nobreakspace}%

\crefname{figure}{\g{Σχήματος}}{\g{Σχημάτων}}
\Crefname{figure}{\g{Σχή\-μα}}{\g{Σχήματα}}
\crefname{table}{\g{Πίνακας}}{\g{Πίνακες}}
\Crefname{table}{\g{Πίνακα}}{\g{Πίνακες}}
\crefname{enumi}{\g{Στοιχείου}}{\g{Στοιχείων}}
\Crefname{enumi}{\g{Στοιχείο}}{\g{Στοιχεία}}
\Crefname{chapter}{\g{Κεφάλαιο}}{\g{Κεφάλαια}}
\crefname{section}{\g{Ενότητας}}{\g{Ενοτήτων}}
\Crefname{section}{\g{Ενότητα}}{\g{Ενότητες}}
\crefname{subsection}{\g{Ενότητας}}{\g{Ενοτήτων}}
\Crefname{subsection}{\g{Ενότητα}}{\g{Ενότητες}}
\Crefname{appendix}{\g{Παράρτημα}}{\g{Παραρτήματα}}


% Front matter
%\frontmatter

% r.1 blank page
%\blankpage

% r.3 full title page
\makeatletter
\renewcommand{\maketitle}{%
	\newpage
	\global\@topnum\z@% prevent floats from being placed at the top of the page
	\begingroup
	\setlength{\parindent}{0pt}%
	\setlength{\parskip}{4pt}%
	\let\@@title\@empty
	\let\@@author\@empty
	\let\@@date\@empty
	\thispagestyle{empty}
	\begin{fullwidth}
		\vfill
		\begin{center}
			\href{https://www.auth.gr/}{\includegraphics[width=.8\textwidth]{auth_logo_text}}\par
			\vspace{1cm}
			\LARGE\textsc{Διπλωματικη Εργασια}\par
			\vspace{6ex}
			\hrule
			\vspace{4ex}
			\Huge\textbf{Επικοινωνία υποσυστημάτων στον νανοδορυφόρο AcubeSAT}\\[1ex]
			\vspace{2.7ex}
			\hrule

			\vspace{4ex}

			\Large
			\begin{tabular}{ll}
				\emph{Συγγραφέας:} & \href{https://github.com/toniker}{Αντώνιος \textsc{Κερεμιδης} \normalsize (\fontfamily{pplx}\selectfont 9717)}
				\\[1.5ex]
				\emph{Επιβλέπων:} & \href{http://ee.auth.gr/en/school/faculty-staff/electronics-computers-department/hatzopoulos-alkiviadis/}{Καθ. Αλκιβιάδης \textsc{Χατζοπουλος}}
			\end{tabular}

			\vspace{6ex}

			\large \textit{Η διπλωματική εργασία κατατίθεται για την \\ εκπλήρωση των υποχρεώσεων για λήψη διπλώματος}\\[0.3cm] % University requirement text
			\textit{στην}\\[0.4cm]
			\href{https://www.eng.auth.gr/gr/archiki.html}{Πολυτεχνική Σχολή}
			\\
			\href{https://ee.auth.gr/}{Τμήμα Ηλεκτρολόγων Μηχανικών \& Μηχανικών Υπολογιστών}
			\\[1cm] % Research group name and department name

			\vfill

			{\large Date}\\[4cm] % Date % TODO

		\end{center}
		\vfill
	\end{fullwidth}
	\endgroup
	\thispagestyle{plain}% suppress the running head
	\tuftebreak% add some space before the text begins
	\@afterindentfalse\@afterheading% suppress indentation of the next paragraph
}
\makeatother
\maketitle

% v.4 copyright page
\newpage
\begin{fullwidth}
~\vfill
\thispagestyle{empty}
\setlength{\parindent}{0pt}
\setlength{\parskip}{\baselineskip}
Copyright \copyright\ \the\year\ Αντώνιος Κερεμίδης

\par\smallcaps{Δημοσιευτηκε απο το \thanklesspublisher}

\justify

\par Αυτή η εργασία χορηγείται με άδεια Creative Commons Αναφορά Δημιουργού 4.0 Διεθνές (CC BY 4.0 --- η ``Άδεια'')· το κείμενο του παρόντος δεν επιτρέπεται να χρησιμοποιηθεί παρά μόνο με βάση την Άδεια. Για να δείτε ένα αντίγραφο αυτής της άδειας, επισκεφτείτε το
\url{https://creativecommons.org/licenses/by/4.0/legalcode.el}, ή δείτε μια "αναγνώσιμη από άνθρωπο" σύνοψη στο \url{https://creativecommons.org/licenses/by/4.0/deed.el}.\index{license}

\par Η εργασία ετοιμάστηκε χρησιμοποιώντας \LaTeX{} με το πρότυπο \href{https://ctan.org/pkg/tufte-latex?lang=en}{\texttt{tufte-latex}} και τις βελτιώσεις του \href{https://github.com/lalider/tufte-latex-thesis}{\texttt{tufte-latex-thesis}}.

\par Το AcubeSAT project εκτελείται με την υποστήριξη του Education Office του \href{https://www.esa.int/}{Ευρωπαϊκού Οργανισμού Διαστήματος}, στα πλαίσια του \href{https://www.esa.int/Education/CubeSats_-_Fly_Your_Satellite/}{προγράμματος Fly Your Satellite!}

\par Οι απόψεις που εκφράζονται στο παρόν από τους συγγραφείς δεν μπορούν \smallcaps{σε καμια περιπτωση να θεωρηθει πως εκφραζουν} την επίσημη άποψη, ή υποστήριξη, του Ευρωπαϊκού Οργανισμού Διαστήματος.

\end{fullwidth}

% r.5 contents
\tableofcontents

\begin{fullwidth}
\listoffigures

\listoftables

\chapter*{Μεταφράσεις ξενόγλωσσης ορολογίας}

\bgroup
\setlength\parskip{.8ex}
% \printacronyms[include=glossary,template=glossary]
\egroup

%\chapter*{List of Acronyms}
%\acuseall%
\bgroup
\setlength\parskip{1ex}
% \printacronyms[pages={display=all,seq/use=false},exclude = {glossary},name = {Ακρωνύμια}]
\egroup

\end{fullwidth}

% r.9 introduction
\cleardoublepage

\chapter*{Περίληψη}

\begin{fullwidth}
	\centering
	\begin{minipage}{107mm}
		\justify
		\g{Η επικοινωνία των υποσυστημάτων σε ένα νάνοδορυφόρο αποτελεί έναν κρίσιμο παράγοντα για την επιτυχία της διαστημικής αποστολής. Σε αυτό το πλαίσιο, η παρούσα διπλωματική εργασία εστιάζει στην χρήση του }CAN Bus \g{ως πρωτοκόλλου επικοινωνίας για τη σύνδεση και αλληλεπίδραση υποσυστημάτων σε έναν νάνοδορυφόρο μεγέθους} 3U (Units)\g{. Αναλύοντας το υλικό, το πρωτόκολλο και το λογισμικό που υλοποιήθηκε, αυτή η εργασία θα εξετάσει τις τεχνικές, τις προκλήσεις και τις λύσεις που εφαρμόστηκαν για τη διασφάλιση αξιόπιστης επικοινωνίας μεταξύ των υποσυστημάτων ενός τέτοιου δορυφόρου. Η εργασία έγινε στα πλάσια υλοποίησης του νανοδορυφόρου AcubeSAT, η οποία υλοποιείται από φοιτητές, κυρίως του Αριστοτελείου Πανεπιστημίου Θεσσαλονίκης. Στην περίπτωση του νανοδορυφόρου} AcubeSAT \g{το} CAN Bus \g{χρησιμοποιείται για την μεταφορά εντολών, δεδομένων πειράματος, και την ανίχνευση σφαλμάτων μεταξύ υποσυστημάτων.}
    \end{minipage}
\end{fullwidth}

\chapter*{Abstract}

\chapter*{Ευχαριστίες}

\g{%
% TODO
}

\mainmatter

\chapter{Εισαγωγή}
\section{Επιστημονική Περιοχή}
\section{Σκοπός και συνεισφορά της διπλωματικής}
\section{Διάρθρωση και Δομή}

\chapter{Ο νανοδορυφόρος AcubeSAT}
\section{Εισαγωγή}
Ο νανοδορυφόρος AcubeSAT δημιουργήθηκε από την φοιτητική ομάδα SpaceDot, υπό την αιγίδα του προγράμματος Fly Your Satellite! 3, του εκπαιδευτικού γραφείου του Ευρωπαϊκού Οργανισμού Διαστήματος. Η ομάδα εδρεύει στο Αριστοτέλειο Πανεπιστήμιο Θεσσαλονίκης, το οποίο παρέχει χώρους εργασίας και χρηματοδότηση, ενώ παράλληλα οι καθηγητές του συνδράμουν στην προσπάθεια της ομάδας. Το project απασχολεί περίπου 90 φοιτητές, οι οποίοι φοιτούν κατά κύριο λόγο φοιτούν στο ΑΠΘ. Ο AcubeSAT είναι ένας νανοδορυφόρος τύπου CubeSat, μεγέθους 3U, με διαστάσεις ($x$ εκατοστά) και βάρος $y$ γραμμάρια. Ένα CubeSat συνήθως δημιουργείται ως δευτερεύον payload ενός δορυφόρου, και \emph{αφήνεται} σε τροχιά γύρω από την Γη. Το CubeSat θα λειτουργεί εντελώς αυτόνομα και δεν θα είναι συνδεδεμένο με κανένα τρόπο με το δορυφόρο μετά την έναρξη λειτουργίας του. Καθώς ένα CubeSat δεν έχει τρόπο για να επηρεάσει την τροχιά του, παρά μόνο τον προσανατολισμό του, είναι απαραίτητο εξ'αρχής να τεθεί σε τροχιά που καλύπτει τις ανάγκες του. Στην περίπτωση του AcubeSAT, αυτή η τροχιά θα είναι σε ύψος περίπου 500 χλμ πάνω από την επιφάνεια της Γης. Η τροχιά θα έχει αρκετή ταχύτητα ώστε να διαρκέσει ενάμιση χρόνο προτού ο νανοδορυφόρος πέσει στην ατμόσφαιρα και καταστραφεί λόγω θερμότητας κατά την επανείσοδο.

\section{Αποστολή}
Ο σκοπός του AcubeSAT είναι να μελετήσει την ανάπτυξη των ευκαρυωτικών κυττάρων του \emph{Saccharomyces cerevisiae} σε συνθήκες διαστήματος. Ο παραπάνω οργανισμός είναι ένας μύκητας, ο οποίος ανήκει %. 
Χρησιμοποιώντας την ομοιότητα που εμφανίζει με τα ανθρώπινα κύτταρα μπορούμε να εξάγουμε συμπεράσματα για την ανάπτυξη των κυττάρων σε μικροσκοπικό επίπεδο κατά τη διάρκεια της ζωής στο διάστημα. Όντας έξω από την προστατευτική ατμόσφαιρα της Γης, τα κύτταρα δέχονται μεγαλύτερες δόσεις ακτινοβολίας, που προέρχεται από %.
Συγκεκριμένα, σκοπός είναι να μελετηθεί η επιρροή της έλλειψης βαρύτητας και η ακτινοβολία που αναφέρθηκε παραπάνω σε τρείς καλλιέργιες του \emph{Saccharomyces cerevisiae}, συγκρίνοντας την ανάπτυξή τους στο διάστημα σε σύγκριση με τη Γη.
\section{Υποσυστήματα}
Ο νανοδορυφόρος αποτελείται από 4 ξεχωριστά υποσυστήματα, τα οποία στεγάζονται στο πάνω μέρος του.
\subsection{Υποσύστημα Προσδιορισμού και Ελέγχου Προσανατολισμού (\acs{ADCS})}
Το υποσύστημα του \acs{ADCS} είναι υπεύθυνο για τον προσανατολισμό του δορυφόρου όσο αυτός βρίσκεται σε τροχιά. Για να το επιτύχει αυτό, χρησιμοποιούονται δύο μαγνητόμετρα υψηλής ακριβείας. Παρόλο που αρκούν οι μετρήσεις από ένα μόνο μανγητόμετρο, ο δορυφόρος έχει εξοπλιστεί με ένα δευτερεύον για περίπτωση όπου χρειαστεί για λόγους ανίχνευσης και αντιμετώπισης βλαβών. Για λόγους εξοικονόμησης ενέργειας, χρησιμοποιείται η τεχνική \textbf{cold redundancy}, όπου το δευτερεύον μαγνητόμετρο είναι απενεργοποιημένο έως ότου εντοπιστεί βλάβη στο κύριο.

Το \acs{ADCS} λειτουργεί υπό τρία διαφορετικά προφίλ, για να καλύψει τις ανάγκες του δορυφόρου ανά πάσα στιγμή. Μόνο ένα προφίλ είναι ενεργοποιημένο τη φορά, και επιλέγονται αυτόματα ανάλογα με τη θέση και της δραστηριότητα του δορυφόρου.
\begin{enumerate}
    \item \textbf{Nadir Pointing}: Ο δορυφόρος στρέφει τη πλευρά +X ώστε να κοιτάζει προς τη Γη. Αυτή η κατεύθυνση στρέφει την κατευθυντική κεραία του δορυφόρου προς την κατάλληλη κατεύθυνση για να εξασφαλίσει ζεύξη, όταν αυτός βρίσκεται πάνω από τον σταθμό βάσης. 
    \item \textbf{Sun Pointing}: Ο δορυφόρος στρέφεται προς τον ήλιο με κατάλληλο τρόπο, ώστε να μεγιστοποιηθεί η παραγωγή ενέργειας από τα ηλιακά πάνελ. Αυτό το προφίλ ενεργοποιείται ανάμεσα σε περάσματα από τον σταθμό βάσης, έτσι ώστε να εξασφαλιστεί θετικό ισοζύγιο ενέργειας στις μπαταρίες.
    \item \textbf{Detumbling}: Ο δορυφόρος κάνει τις κατάλληλες κινήσεις για να φτάσει τη γωνιακή του ταχύτητα στο μηδέν. Αυτό επιτυγχάνεται με χρήση ενός απλού αλγορίθμου αυτομάτου ελέγχου και τις μετρήσεις του μαγνητομέτρου. Αυτή η λειτουργία ενεργοποιείται όταν ο δορυφόρος δεν έχει άμεση ανάγκη να βρίσκεται σε κάποιο από τα άλλα προφίλ. Η διατήρηση της γωνιακής ταχύτητας κοντά στο μηδέν, βοηθά την ανάκτηση ελέγχου όταν χρειαστεί να προσανατολιστεί ο δορυφόρος σε άλλη κατεύθυνση. Ένα επιπλέον πλεονέκτημα είναι ότι βοηθά στην εξασφάλιση ραδιοζεύξης με τη χρήση της % μακριας κεραιας
\end{enumerate}
\subsection{Υποσύστημα Τηλεπικοινωνιών (\acs{COMMS})}
Το υποσύστημα τηλεπικοινωνιών είναι υπεύθυνο για την επικοινωνία με το σταθμό βάσης, όσο ο δορυφόρος βρίσκεται σε τροχιά. Για να επιτύχει τους σκοπούς του, ο δορυφόρος χρησιμοποιεί δύο είδη κεραιών. Μία από τις κεραίες του δορυφόρου λειτουργεί στο S-Band εύρος συχνοτήτων. Ο σκοπός αυτής της κεραίας είναι η μεταφορά των δεδομένων των επιστημονικών πειραμάτων σε μορφή φωτογραφιών PNG. Καθώς αυτή η κεραία είναι κατευθυντική απαιτεί οπτική επαφή με τον σταθμό βάσης. Η δεύτερη κεραία διπόλων χρησιμεύει στην συνεχή επικοινωνία με τον δορυφόρο, χρησιμοποιώντας την μπάντα UHF. Χάρη στη χαμηλή ταχύτητα μεταφοράς δεδομένων που προσφέρεται από το φάσμα των UHF, η επικοινωνία με τον δορυφόρο περιορίζεται σε βασικές τηλε-εντολές και τηλεμετρία.

Για τις διεργασίες του υποσυστήματος τηλεπικοινωνιών χρησιμοποιείται η πλακέτα SatNOGS COMMS της Libre Space Foundation. Όλο το υποσύστημα έχει υλοποιηθεί με βάση τα τηλεπικοινωνιακά πρότυπα CCSDS.

\subsection{Υποσύστημα Διαχείρισης Δεδομένων (\acs{OBDH})}
Το υποσύστημα διαχείρισης δεδομένων είναι υπεύθυνο για τον σχεδιασμό των διεπαφών δεδομένων εντός του δορυφόρου. Ένα σημαντικό κατόρθωμα του υποσυστήματος, αλλά και της ευρύτερης ομάδας, είναι ο σχεδιασμός και υλοποίηση της πλακέτας που στεγάζει τα υποσυστήματα του \textbf{On-Board Computer} και του \textbf{Attitude Determination and Control Subsystem}.

% PCB pic
Η πλακέτα που δημιουργήθηκε ακολουθεί το πρότυπο συνδεσιμότητας PC/104 και περιλαμβάνει μνήμες MRAM, NAND και τον μικροελεγκτή που στεγάζει το OBC στην επάνω πλευρά του. Στην κάτω πλευρά της πλακέτας στεγάζεται ο μικροελεγκτής για το υποσύστημα του \textbf{ADCS}, μαζί με το μαγνητόμετρο και τα δύο γυροσκόπια που απαιτεί το υποσύστημα.

Το υποσύστημα του OBC είναι υπεύθυνο για τον μηχανισμό ανίχνευσης, απομόνωσης και αντιμετώπισης βλαβών σε επίπεδο υποσυστημάτων για τον υπόλοιπο δορυφόρο. Χρησιμοποιώντας μηνύματα μέσω του \textbf{CAN Bus}, μελετά τις παραμέτρους που αναφέρουν τα υπόλοιπα υποσυστήματα. Σε οποιοδήποτε σφάλμα που ανιχνεύεται στην επικοινωνία ή στα δεδομένα που λαμβάνονται, το OBC είναι υπεύθυνο για την επαναφορά του προβληματικού υποσυστήματος. Η επαναφορά αυτή γίνεται με εντολές στο υποσύστημα \textbf{Electrical Power Supply} για \emph{power cycle}, το οποίο παρέχεται από την εταιρεία \emph{ISISPACE}.

\subsection{Υποσύστημα επιστημονικής μονάδας (\acs{SU})}

Το υποσύστημα επιστημονικής μονάδας (SU) κατέχει περίπου τα 2/3 του όγκου στο δορυφόρο. Για την διεξαγωγή του πειράματος, το \acs{SU} περιλαμβάνει μία πλακέτα η οποία στεγάζει τον μικροελεγκτή που χρησιμοποιείται για τον έλεγχο των επιστημονικών οργάνων. Το επιστημονικό πείραμα απαιτεί την φωτογράφιση των φωσφορίζων κυττάρων σε τακτά χρονικά διαστήματα της ανάπτυξής τους. Για τον σκοπό αυτό χρησιμοποιείται ένας πρωτοπόρος συνδυασμός από ένα Lab-on-a-chip, με μία κάμερα που καταγράφει την ανάπυτξη των κυττάρων του \textit{Saccharomyces cerevisiae}.

Το πείραμα απαιτεί τον δικό του υποστηρικτικό εξοπλισμό, και ένα από αυτά είναι heaters για διαχείριση της θερμοκρασίας. Καθώς η θερμοκρασία του δορυφόρου κατά τη διάρκεια της τροχιάς του θα μεταβάλλεται τρείς αισθητήρες θερμοκρασίας υψηλής ακριβείας θα δίνουν την είσοδο στο αυτόματο σύστημα ελέγχου, έτσι ώστε η θερμοκρασία του Lab-on-a-chip να διατηρείται σταθερή στους $30^o C \pm 2^oC$. % ref ddjf_pl 

Παραπάνω, η φωτογράφιση των κυττάρων απαιτεί 
\section{Επικοινωνία των υποσυστημάτων}

Όπως αναφέρθηκε παραπάνω, μία από τις αρμοδιότητες του υποσυστήματος OBDH είναι ο σχεδιασμός των διεπαφών δεδομένων εντός του δορυφόρου. 
Αναλύοντας πρωτόκολλα για επικοινωνία εντός του δορυφόρου, υπάρχουν πολλές επιλογές που αρμόζουν στις ανάγκες του AcubeSAT. Τα πρωτόκολλα που εξετάστηκαν είναι τα παρακάτω:
\begin{itemize}
	\item UART
	\item SPI
	\item I2C
	\item CAN Bus
	\item SpaceWire
\end{itemize}

Από αυτά τα πρωτόκολλα επιλέχθηκε το \acs{CAN}, για τρείς βασικούς λόγους. Ο πρώτος λόγος είναι η ισχυρή αξιοπιστία του πρωτοκόλλου, ο δεύτερος είναι η άμεση επικοινωνία των υποσυστημάτων χρησιμοποιώντας ένα κοινό μέσο και ο τρίτος είναι η υποστήριξη του πρωτοκόλλου από τους κατασκευαστές των μικροελεγκτών του δορυφόρου, Microchip και STM.

\section{CAN Bus}
\subsection{Ιστορία}
Το Controller Area Network (CAN) είναι ένα εύρωστο πρωτόκολλο σειριακής επικοινωνίας που αναπτύχθηκε από τη Bosch τη δεκαετία του 1980 για να παρέχει ένα σύστημα επικοινωνίας χαμηλού κόστους και υψηλής ταχύτητας μεταξύ των ηλεκτρονικών μονάδων ελέγχου (ECU) στα οχήματα. Πλέον είναι ευρέως διαδεδομένο στην αυτοκινητοβιομηχανία και την αεροδιαστημική βιομηχανία, καθώς και στον βιομηχανικό αυτοματισμό και τον ιατρικό εξοπλισμό. Ο δίαυλος CAN επιτρέπει σε πολλαπούς κόμβους να επικοινωνούν μεταξύ τους μέσω ενός μόνο καλωδίου συνεστραμμένου ζεύγους, χρησιμοποιώντας ένα πρωτόκολλο που βασίζεται σε μηνύματα. Είναι γνωστό για τις δυνατότητες ανίχνευσης και διόρθωσης σφαλμάτων, καθιστώντας το ιδανικό για εφαρμογές κρίσιμες για την ασφάλεια.

Ο International Organization for Standardization (ISO) έχει διαδραματίσει σημαντικό ρόλο στην ανάπτυξη και την τυποποίηση του CAN. Το 1993, ο ISO κυκλοφόρησε το πρώτο πρότυπο για το CAN, γνωστό ως ISO 11898. Αυτό το πρότυπο καθόρισε τα επίπεδα υλικού και ζεύξης δεδομένων του πρωτοκόλλου. Αυτά τα επίπεδα περιλαμβάνουν τον χρονισμό bit, το πλαίσιο μηνυμάτων και το μηχανισμό ανίχνευσης και διόρθωσης σφαλμάτων. Έκτοτε, το ISO συνέχισε να ενημερώνει και να επεκτείνει το πρότυπο CAN, με την πιο πρόσφατη έκδοση να είναι το ISO 11898-1:2015. Αυτά τα πρότυπα έχουν βοηθήσει στη διασφάλιση της διαλειτουργικότητας και της συμβατότητας μεταξύ των διαφορετικών υλοποιήσεων του πρωτοκόλλου CAN. Στο πρωτόκολλο ISO 11898-1:2015 βασίζεται και η υλοποίηση του περιφερειακού CAN στον μικροελεγκτή ATMEL SAMV71Q21B που χρησιμοποιείται στον δορυφόρο AcubeSAT.
\subsection{Χρήση}
\subsection{Το επίπεδο υλικού}
\subsection{CAN-FD}

\chapter{Σχεδίαση του νανοδορυφόρου}
Κατά τη διάρκεια της σχεδίασης του νανοδορυφόρου, η ομάδα κλήθηκε να ερευνήσει την εφικτότητα του πλάνου για την επικοινωνία των υποσυστημάτων και να αναπτύξει λεπτομερρείς περιγραφές για το πώς θα επιτευχθεί αυτό. Στην παρούσα ενότητα θα παρουσιαστούν οι αποφάσεις που πάρθηκαν και οι λόγοι που οδήγησαν σε αυτές. 
\section{Έρευνα data link budget}
Το πρώτο πρόβλημα που αντιμετώπισε η ομάδα είναι 
\subsection{Ορισμός AcubeSAT specific πρωτοκόλλου CAN}
Το δεύτερο πρόβλημα που αντιμετώπισε η ομάδα είναι η επιλογή της μορφής των μηνυμάτων που θα μεταφέρονται μέσω του CAN. Αποφαστίστηκε ότι αυτά τα μηνύματα θα ακολουθούν πρωτόκολλο επιπέδου Application Layer (OSI Layer 7). Η ομάδα αρχικά εξέτασε επιλογές που είναι ήδη εδραιωμένες στον τομέα, όπως το SpaceCAN, το CANOpen ή το ECSS-E-ST-50-15C. Τα παραπάνω πρωτόκολλα δεν ικανοποιούν τις απαιτήσεις της ομάδας για χαμηλή πολυπλοκότητα. Επιπλέον, εφ'όσον αυτή τη στιγμή ακόμα δεν υπάρχει πρότυπο για πρωτόκολλο σε CubeSat, η ομάδα αποφάσισε να σχεδιάσει εξ'αρχής ένα νέο πρωτόκολλο το οποίο θα καλύπτει τις ανάγκες της αποστολής.

Οι λεπτομέρειες του πρωτοκόλλου φαίνονται στο έγγραφο DDJF\_OBDH.pdf, στο παράρτημα Α. Για ευκολία στον αναγνώστη, παρακάτω εμφανίζεται ένα διάγραμμα που περιγράφει την αρχιτεκτονική ενός μηνύματος.

% insert figure of a tp message here.

\chapter{Υλοποίηση}
% Open Source Project
% Η συνεισφορά μου είναι MIT Licensed, MR Links
% Ειναι reviewed απο συμφοιτητες
\section{Περιβάλλον Υλοποίησης}

Η παρούσα εργασία έγινε κατά τη διάρκεια μου ως μέλος του υποσυστήματος OBC, κατά τη διάρκεια του εαρινού εξαμήνου του έτους 2022. Καθώς η ομάδα ήταν στο στάδιο σχεδίασης της πλακέτας που στεγάζει τα υποσυστήματα των OBC και ADCS, η ανάπτυξη έγινε σε δύο πλακέτες \textbf{SAM V71 XPLAINED ULTRA EVALUATION KIT}. Οι πλακέτες αυτές παράγονται από την Microchip και περιέχουν τον ίδιο μικροελεγκτή που θα χρησιμοποιηθεί στα τρία υποσυστήματα του νανοδορυφόρου. Η επιλογή αυτή έγινε για να επιταχυνθεί η ανάπτυξη του λογισμικού, καθώς η πλακέτα περιέχει τον απαραίτητο transceiver για την λειτουργία του CAN Bus. 

Όπως αναφέρθηκε στην παραπάνω ενότητα, για την επικοινωνία στο δίαυλο απαιτούνται δύο συνδέσεις, CANH και CANL. Το πρωτόκολλο CAN δεν απαιτεί την χρήση αντιστάσεων τερματισμού, όμως η χρήση τους είναι απαραίτητη για την αποφυγή ανακλάσεων στο σήμα. Στην παρούσα περίπτωση, η αποφυγή αυτή επιτυγχάνεται με την χρήση καλωδίων παράλληλης ζεύξης μικρού μήκους, όπως φαίνεται στην παρακάτω εικόνα.
\begin{figure}[h]
	\includegraphics[angle=270,origin=c]{media/images/20220523_194452.jpg}
\end{figure}

Το σύνολο της εργασίας μου έγινε υπό άδεια MIT, όπως λειτουργεί και το υπόλοιπο project του λογισμικού του δορυφόρου. Η άδεια αυτή επιτρέπει την οποιαδήποτε χρήση του λογισμικού, χωρίς άδεια από τον δημιουργό. Ο δημιουργός όμως δεν παρέχει καμία εγγύηση, αλλά και καμία υποχρέωση καλής λειτουργίας του λογισμικού. 

Κατά τη διάρκεια της ανάπτυξης του λογισμικού η δουλειά μου καταγράφθηκε χρησιμοποιώντας το Git για Version Control στα αποθετήρια της ομάδας. Οι παρακάτω σύνδεσμοι οδηγούν στα σχετικά Merge Requests που έγιναν στο Gitlab της ομάδας. Εκεί φαίνεται ο χαρακτήρας της ομάδας, όσο αφορά το peer-review. Όλες οι αλλαγές που έγιναν στον κώδικα ελέγχθηκαν από τουλάχιστον δύο συμφοιτητές, πριν ενσωματωθούν στον κώδικα του δορυφόρου. 

\section{Περιορισμοί συστήματος} % static memory allocation, low ram etc

Ο χαρακτήρας του συστήματος που αναπτύχθηκε είναι αυστηρά περιορισμένος από τους πόρους του μικροελεγκτή. Ο μικροελεγκτής που χρησιμοποιήθηκε έχει μόλις 384KB από στατική RAM διαθέσιμη για όλες τις λειτουργίες του κάθε υποσυστήματος. Περαιτέρω, για να επιτευχθούν στόχοι αξιοπιστίας η ομάδα αποφάσισε να απαγορεύσει την δυναμική εκχώρηση μνήμης στο λογισμικό του δορυφόρου. Οι δύο αυτοί περιορισμοί συνδράμουν στην δυσκολία της υλοποίησης, καθώς η ανάπτυξη γίνεται με την χρήση της γλώσσας C++, η οποία δεν παρέχει εύκολη υποστήριξη για την διαχείριση της μνήμης. Από την αρχή της ανάπτυξης του λογισμικού, η ομάδα επέλεξε να χρησιμοποιήσει την βιβλιοθήκη Embedded Template Library (ETL), η οποία προσφέρει συναρτήσεις ανάλογες της C++ Standard Library (STL). Η σημαντική διαφοροποίηση στις δύο βιβλιοθήκες είναι ότι η ETL δίνει τη δυνατότητα χρήσης αποκλειστικά στατικά εκχωρημένης μνήμης. Για να το επιτύχει αυτό, ο προγραμματιστής πρέπει να ορίσει το μέγιστο μέγεθος για όλους τους τύπους κοντέινερ. Για παράδειγμα, ένα μήνυμα πρέπει να έχει αυστηρά ορισμένο μέγεθος στον πίνακα των δεδομένων του, και μία ουρά πρέπει να έχει ορισμένο το μέγιστο μέγεθος της. Αυτός ο περιορισμός επιβάλλει την χρήση της μνήμης με πολύ προσεκτικό τρόπο, καθώς η χρήση μνήμης που δεν έχει αναλογιστεί μπορεί να οδηγήσει σε απρόβλεπτες συμπεριφορές του προγράμματος. Παρόλα αυτά, η ομάδα κατάφερε να υλοποιήσει ένα σύστημα που ικανοποιεί τις απαιτήσεις του δορυφόρου, χωρίς να θυσιάσει την αξιοπιστία του. 
\section{Η διαδρομή ενός μηνύματος}
\section{Η μεταφορά μηνυμάτων ως πακέτα}
\section{Ουρές μηνυμάτων με βάση το αναγνωριστικό του αποστολέα}
\section{Διαφοροποίηση στα CAN-TP και non-CAN-TP μηνύματα}
\section{Αποστολή μεγάλων μηνυμάτων}
\section{Αποστολή φωτογραφιών του πειράματος}

\chapter{Παραδείγματα χρήσης της υλοποίησης}
\section{Χρήση στο Environmental Testing Campaign του OBC/ADCS Board}
	\subsection{Μεταφορά παραμέτρων}
	\subsection{Διακοπή λειτουργίας σε υψηλή θερμοκρασία}
	\subsection{Εφαρμογή λύσης και επικύρωση ορθής λειτουργίας}
\section{Χρήση στο Environmental Testing Campaign του Payload}

%\backmatter
\appendix

\begin{fullwidth}
\bgroup
% \printbibliography[heading=bibnumbered,title={Βιβλιογραφία}]
\egroup
\end{fullwidth}

\end{document}

