%%
% If they're installed, use Bergamo and Chantilly from www.fontsite.com.
% They're clones of Bembo and Gill Sans, respectively.
%\IfFileExists{bergamo.sty}{\usepackage[osf]{bergamo}}{}% Bembo
%\IfFileExists{chantill.sty}{\usepackage{chantill}}{}% Gill Sans

%\usepackage[protrusion=true,expansion,babel=true]{microtype}

%%
% Just some sample text
\usepackage{lipsum}

%%
% For nicely typeset tabular material
\usepackage{booktabs}
\usepackage{multirow}
\usepackage{adjustbox}
\usepackage{array}
\usepackage{color}
\usepackage{tabularray}

\newcolumntype{L}[1]{>{\noindent\RaggedRight\arraybackslash\hspace{0pt}}p{#1}}
\newcolumntype{R}[1]{>{\noindent\RaggedLeft\arraybackslash\hspace{0pt}}p{#1}}
\newcolumntype{C}[1]{>{\noindent\Centering\arraybackslash\hspace{0pt}}p{#1}}

%%
% For graphics / images
\usepackage{graphicx}
\setkeys{Gin}{width=\linewidth,totalheight=\textheight,keepaspectratio}
\graphicspath{{media/}}

% What
\setcounter{secnumdepth}{3}
\setcounter{tocdepth}{3}

% Bibliography
\usepackage[
style=ieee,
citestyle=numeric-comp,
autocite=inline,
autopunct=true,
backend=biber,
maxbibnames=99,
maxcitenames=2,
mincitenames=1,
]{biblatex}

%%
% Maths
\usepackage{amsmath}

% Prints a trailing space in a smart way.
\usepackage{xspace}

% Inserts a blank page
\newcommand{\blankpage}{\newpage\hbox{}\thispagestyle{empty}\newpage}

% Referencing
\usepackage[unicode]{hyperref}
\usepackage[nameinlink]{cleveref}

\hypersetup{%
	bookmarksnumbered = true, % Show section numbering in the PDF table of contents. Allows easier browsing of the document
	colorlinks = true, % uncomment this line if you prefer colored hyperlinks (e.g., for onscreen viewing)
%	pdfborder = {0 0 0},
	bookmarksdepth = subsection,
%	citecolor = DarkGreen,
%	linkcolor = DarkBlue,
%	urlcolor = DarkGreen,
}

%\AtBeginDocument{%
%	\hypersetup{
%		pdfauthor={\plainauthor}
%		pdftitle={\plaintitle},
%	}%
%}

\usepackage[binary-units=true]{siunitx}

% Modifications to the default tufte template to make it more applicable to a thesis
% Credit goes to Tiffany Tseng, lalider, see https://github.com/lalider/tufte-latex-thesis

\usepackage[parfill]{parskip}

% remove paragraph indentation
\makeatletter
% Paragraph indentation and separation for normal text
\renewcommand{\@tufte@reset@par}{%
	\setlength{\RaggedRightParindent}{0.0pc}%
	\setlength{\JustifyingParindent}{0.0pc}%
	\setlength{\parindent}{0pc}%
	\setlength{\parskip}{\baselineskip}%
}
\@tufte@reset@par

% Paragraph indentation and separation for marginal text
\renewcommand{\@tufte@margin@par}{%
	\setlength{\RaggedRightParindent}{0.0pc}%
	\setlength{\JustifyingParindent}{0.0pc}%
	\setlength{\parindent}{0.0pc}%
	\setlength{\parskip}{10pt}%
}
\makeatother

\titleformat{\section}%
[hang]% shape
{\normalfont\Large}% format applied to label+text
{\thesection}% label
{1em}% horizontal separation between label and title body
{}% before the title body
[]% after the title body

%\titlespacing*{\section}{0pt}{3.5ex plus 1ex minus .2ex}{2.3ex plus .2ex}
%\titlespacing*{\subsection}{0pt}{3.25ex plus 1ex minus .2ex}{1.5ex plus.2ex}
\titlespacing*{\section}{0pt}{30pt}{20pt}
\titlespacing*{\subsection}{0pt}{20pt}{5pt}


% Disable adding empty pages without a purpose (not intended for printing in a book format...)
\makeatletter
\def\cleardoublepage{
	\clearpage%
}
\makeatother

% Acronyms
\usepackage{acro}
\acsetup{
	use-id-as-short,
	make-links=false,
	case-sensitive=false,
	patch/floats=false,
	patch/caption=false,
	patch/tabularx=false
}
\DeclareAcronym{FDIR}{short=FDIR,long={Fault Detection, Isolation and Recovery}, foreign = {Εντοπισμός, Απομόνωση και Διόρθωση Αποτυχιών}}
\DeclareAcronym{ADCS}{short = ADCS, long = {Attitude Determination and Control Subsystem}}
\DeclareAcronym{COMMS}{short = COMMS, long = {Communications}}
\DeclareAcronym{EPS}{short = EPS, long = {Electrical Power Subsystem}}
\DeclareAcronym{OBC}{short = OBC, long = {On-Board Computer}}
\DeclareAcronym{OBDH}{short = OBDH, long = {On-Board Data Handling}}
\DeclareAcronym{OBSW}{short = OBSW, long = {On-Board Software}}
\DeclareAcronym{OPS}{short = OPS, long = {Operations}}
\DeclareAcronym{SYE}{short = SYE, long = {Systems Engineering}}
\DeclareAcronym{SU}{short = SU, long = {Science Unit}}
\DeclareAcronym{EMC}{short = EMC, long = {Electromagnetic Compatibility}, foreign = {Ηλεκτρομαγνητική Συμβατότητα}}
\DeclareAcronym{CDR}{short = CDR, long = {Critical Design Review}}
\DeclareAcronym{GS}{short = GS, long = {Ground Station}, foreign = Σταθμός Βάσης}
\DeclareAcronym{TC}{short = TC, long = Telecommand, foreign = Τηλεεντολές}
\DeclareAcronym{TM}{short = TM, long = Telemetry, foreign = Τηλεμετρία}
\DeclareAcronym{RF}{short = RF, long = RadioFrequency}
\DeclareAcronym{CCSDS}{short = CCSDS, long = {The Consultative Committee for Space Data Systems}}
\DeclareAcronym{ISM}{short = ISM, long = {Industrial, Scientific, Medical}}
\DeclareAcronym{COTS}{short = COTS, long = {Commercial Off-The-Shelf}}
\DeclareAcronym{PCDU}{short = PCDU, long = {Power Conditioning \& Distribution Unit}}
\DeclareAcronym{MPPT}{short = MPPT, long = {Maximum Power Point Tracking}}
\DeclareAcronym{ECSS}{short = ECSS, long = {European Cooperation for Space Standardization}}
\DeclareAcronym{PUS}{short = PUS, long = {Packet Utilisation Standard}}
\DeclareAcronym{UHF}{short = UHF, long = {Ultra-High Frequency}}
\DeclareAcronym{LEO}{short = LEO, long = {Low Earth Orbit}, foreign = {Χαμηλή Γήινη Τροχιά}}
\DeclareAcronym{PDMS}{short = PDMS, long = {Polydimethylsiloxane}, foreign = {Πολυδιμεθυλοσιλοξάνη}, cite = volpetti_microfluidic_biodisplay_2017}
\DeclareAcronym{PA}{short = PA, long = {Product Assurance}}
\DeclareAcronym{PCB}{short = PCB, long = {Printed Circuit Board}, foreign = {Τυπωμένη Πλακέτα}}
\DeclareAcronym{GMAT}{short = GMAT, long = {General Mission Analysis Tool}}
\DeclareAcronym{MRAM}{short = MRAM, long = {Magnetoresistive Random-Access Memory}}
\DeclareAcronym{CAN}{short = CAN, long = {Controller Area Network}}
\DeclareAcronym{CAN-FD}{short = CAN-FD, long = {Controller Area Network with Flexible Data-Rate}}
\DeclareAcronym{CAN-TP}{short = CAN-TP, long = {Controller Area Network Transport Protocol}}
\DeclareAcronym{MCU}{short = MCU, long = {microcontroller}, extra={MicroController Unit}, foreign = {Μι\-κρο\-ε\-λεγ\-κτής}}
\DeclareAcronym{RTOS}{short = RTOS, long = {Real-Time Operating System}, foreign = {Λειτουργικό Σύστημα Πραγματικού Χρόνου}}
\DeclareAcronym{SAVOIR}{short = SAVOIR, long = {Space AVionics Open Interface aRchitecture}}
\DeclareAcronym{YAMCS}{short = YAMCS, long = {Yet Another Mission Control System}}
\DeclareAcronym{COBS}{short = COBS, long = {Consistent Overhead Byte Stuffing}}
\DeclareAcronym{SRAM}{short = SRAM, long = {Static Random Access Memory}}
\DeclareAcronym{I2C}{short = {I\textsuperscript{2}C}, alt = {TWI}, extra = {Two-Wire Interface}, long = {Inter-Integrated Circuit}}
\DeclareAcronym{ETL}{short = ETL, long = {Embedded Template Library}}
\DeclareAcronym{HAL}{short = HAL, long = {Hardware Abstraction Library}}
\DeclareAcronym{IDE}{short = IDE, long = {Integrated Development Environment}}
\DeclareAcronym{USB}{short = USB, long = {Universal Serial Bus}}
\DeclareAcronym{UART}{short = UART, long = {Universal Asynchronous Serial Bus}}
\DeclareAcronym{FMEA}{short = FMEA, long = {Failure Mode and Effects Analysis}}
\DeclareAcronym{FMECA}{short = FMECA, long = {Failure Mode, Effects and Criticality Analysis}}
\DeclareAcronym{HSIA}{short = HSIA, long = {Hardware/Software Interaction Analysis}}
\DeclareAcronym{XTCE}{short = XTCE, long = {XML Telemetric and Command Exchange}}
\DeclareAcronym{RAMS}{short = RAMS, long = {Reliability, Availability, Maintainability and Safety}, foreign = {Αξιοπιστία, Διαθεσιμότητα, Συντηρησιμότητα και Ασφάλεια}}
\DeclareAcronym{MAIV}{short = MAIV, long = {Maintenance, Assembly, Integration and Verification}, foreign = {Συντήρηση, Κατασκευή, Συναρμολόγηση και Επαλήθευση}}
\DeclareAcronym{IC}{short = IC, long = {Integrated Circuit}, foreign = {Ολοκληρωμένο Κύκλωμα}}
\DeclareAcronym{SCL}{short = SCL, long = {Serial Clock}}
\DeclareAcronym{SDA}{short = SDA, long = {Serial Data}}
\DeclareAcronym{SEL}{short = SEL, long = {Single Event Latchup}, foreign = {Μανδάλωση Απλής Προσβολής}}
\DeclareAcronym{SEE}{short = SEE, long = {Single Event Effects}}
\DeclareAcronym{SEFI}{short = SEFI, long = {Single Event Functional Interrupt}, foreign = {Διακοπή Λειτουργίας Απλής Προσβολής}}
\DeclareAcronym{GNSS}{short = GNSS, long = {Global Navigation Satellite System}, foreign = {Παγκόσμιο Δορυφορικό Σύστημα Πλοήγησης}}
\DeclareAcronym{HTTP}{short = HTTP, long = {Hypertext Transfer Protocol}}
\DeclareAcronym{API}{short = API, long = {Application Programming Interface}, foreign = {Διεπαφή Προγραμματισμού Εφαρμογών}}
\DeclareAcronym{RAM}{short = RAM, long = {Random Access Memory}, foreign = {Μνήμη Τυχαίας Προσπέλασης}}
\DeclareAcronym{ROM}{short = ROM, long = {Read-Only Memory}, foreign = {Μνήμη Μόνο Ανάγνωσης}}
\DeclareAcronym{PMON}{short = PMON, long = {Parameter Monitoring Definition}, foreign = {Ορισμός Παρακολούθησης Παραμέτρου}}
\DeclareAcronym{DDJF}{short = DDJF, long = {Design Definition and Justification File}}
\DeclareAcronym{DOA}{short = DOA, long = {Dead On Arrival}}
\DeclareAcronym{TMR}{short = TMR, long = {Triple Modular Redundancy}}
\DeclareAcronym{LED}{short = LED, long={Light-Emitting Diode}}
\DeclareAcronym{GCC}{short = GCC, long={GNU C Compiler}}
\DeclareAcronym{MPU}{short = MPU, long={Memory Protection Unit}}
\DeclareAcronym{PNG}{short = PNG, long={Portable Network Graphics}}
\DeclareAcronym{SWD}{short = SWD, long={Serial Wire Debug}}
\DeclareAcronym{PWM}{short = PWM, long={Pulse Width Modulation}}
\DeclareAcronym{LCL}{short = LCL, long={Latchup Current Limiter}}
\DeclareAcronym{DAC}{short = DAC, long={Digital-to-Analog Converter}}
\DeclareAcronym{ID}{short = ID, long={Identifier}}
\DeclareAcronym{SPI}{short = SPI, long={Serial Peripheral Interface}}
\NewDocumentCommand\draft{m}{%
	\textcolor[HTML]{bf616a}{#1}%
}

\makeatletter
\newcommand\footurl@[1]{\footnote{\url@{#1}}}
\DeclareRobustCommand{\footurl}{\hyper@normalise\footurl@}

\newcommand\foothref@[2]{\href@{#1}{#2}\hyper@punct\footnote{\url@{#1}}}
\def\Hy@foothref#{%
	\hyper@normalise\foothref@
}
\DeclareRobustCommand*{\foothref}[1][]{%
	\begingroup%
	\global\def\hyper@punct{#1}%
	\@ifnextchar\bgroup\Hy@foothref{\hyper@normalise\foothref@}%
}
\makeatother

\newcommand\numberthis{\addtocounter{equation}{1}\tag{\theequation}}

% Transitions
\AtBeginDocument{
\colorlet{invalid}{MaterialAmber900}
\colorlet{unexpected}{MaterialRed900}
}

\def\unchecked{\textcolor{MaterialGrey800}{\texttt{Unchecked}}}
\def\ok{\texttt{OK}}
\def\unexpected{\textcolor{unexpected}{\texttt{Unexpected\_Value}}}
\def\hilim{\textcolor{unexpected}{\texttt{Above\_High\_Limit}}}
\def\lolim{\textcolor{unexpected}{\texttt{Below\_Low\_Limit}}}
\def\invalid{\textcolor{invalid}{\texttt{Invalid}}}
\def\ar{ \( \rightarrow \) }

\makeatletter

\makeatother

\NewAcroTemplate[list]{glossary}{%
	\begin{description}
		\acronymsmapF{%
			\item[\sffamily\textbf{\acrowrite{short}}] \acrowrite{long}%
			\acroifanyT {foreign,extra} {~(}%
			\acrowrite {foreign}%
			\acroifallT {foreign,extra} {,~}%
			\acrowrite {extra}%
			\acroifanyT {foreign,extra} {)}%
			\acropagefill%
			\acropages%
				{ \acrotranslate {page} \nobreakspace }%
				{ \acrotranslate {pages} \nobreakspace }%
		}%
		{ \item \AcroRerun{list} }%
		\end {description}
	}

% Various utilities
\usepackage{tabularx}
\usepackage[singlelinecheck=false]{caption}
\usepackage{subcaption}
\usepackage{comment}

% Generates the index
\usepackage{makeidx}
\makeindex

\usepackage{tikz}
\usepackage{pgfplots}
\pgfplotsset{compat = 1.3}
\usetikzlibrary{pie}
\usepackage{physics}
\let\Re=\relax
\let\Im=\relax

% Colours
\definecolor{off}{HTML}{e53935}
\definecolor{on}{HTML}{43a047}
\usepackage{xcolor-solarized}
\usepackage{xcolor-material}
\usepackage{colortbl}

%\usepackage{todonotes}
\usepackage{enumitem}
\usepackage[absolute,overlay]{textpos}

% Code
\usepackage{minted}
\usepackage{xpatch,letltxmacro}
\LetLtxMacro{\cminted}{\minted}
\let\endcminted\endminted
\xpretocmd{\cminted}{\RecustomVerbatimEnvironment{Verbatim}{BVerbatim}{}}{}{}
\definecolor{mintedbg}{rgb}{1,1,1}
\setminted{
	bgcolor=mintedbg,
	tabsize=2,
	fontsize=\footnotesize,
	linenos,
	breaklines
}
\setminted[text]{baselinestretch=0.8}

% Git magic
\makeatletter
\write18{git log --pretty='format:\@percentchar Creset\@percentchar s' --no-merges -1 > \jobname.git1.tmp}
\write18{git rev-parse --short HEAD > \jobname.git2.tmp}
\def\gitcommit{\input{\jobname.git2.tmp}\unskip}
\def\gitcommitmessage{\input{\jobname.git1.tmp}\unskip}
\makeatother

% Smart citations

\makeatletter

%\def\blx@imc@mkbibquote{\blx@enquote}

\newbibmacro*{cite:title}{%
  \printtext[bibhyperref]{%
    \printfield[citetitle]{labeltitle}}%
  \finentry}

\newbibmacro*{cite:shorthand}{%
	\printtext[bibhyperref]{\printfield{shorthand}}}

\newbibmacro*{cite:full}{%
	\iffieldundef{shorthand}
	{\printnames{labelname}%
		\setunit*{\printdelim{nametitledelim}}%
		\usebibmacro{cite:title}}%
	{\usebibmacro{cite:shorthand}}}

\newbibmacro*{cite:marginnote}{%
	\marginnote{%
		% \mkbibbrackets should be used here probably
		[%
			\printfield{labelprefix}%
			\printfield{labelnumber}%
		]
		\usebibmacro{cite:full}%
	}%
}

\DeclareCiteCommand{\dualcite}[\mkbibbrackets]
  {\usebibmacro{cite:init}%
   \usebibmacro{prenote}}
  {\usebibmacro{citeindex}%
   \usebibmacro{cite:marginnote}%
   \usebibmacro{cite:comp}}%
  {}
  {\usebibmacro{cite:dump}%
   \usebibmacro{postnote}}

\def\tuftecite{
	\iffootnote{footnote}{text}
}

% Patching commands that enter information in margins, so that we can detect which
% citation should be used
\newtoggle{inmargin}
\pretocmd{\marginpar}{\toggletrue{inmargin}}{}{}
\apptocmd{\@xympar}{\togglefalse{inmargin}}{}{}
\pretocmd{\@tufte@float}{\toggletrue{inmargin}}{}{}
\pretocmd{\end@tufte@float}{\togglefalse{inmargin}}{}{}
\pretocmd{\@tufte@margin@float}{\toggletrue{inmargin}}{}{}
\pretocmd{\end@tufte@margin@float}{\togglefalse{inmargin}}{}{}

\def\autocite{%
	\iftoggle{inmargin}\parencite\dualcite%
}

% Tufte disables subparagraphs. This brings them back.
\renewcommand\subparagraph{\@startsection{subparagraph}{5}{\parindent}%
	{3.25ex \@plus1ex \@minus .2ex}%
	{-1em}%
	{\hspace{1em}\normalfont\normalsize\itshape}}

% \acuse command that also adds the current page
%\NewAcroTemplate{fdirthesisempty}{}
%\NewAcroCommand\acusepage{m}{}
\let\acusepage\acuse

\makeatother